% ---------------------------------------------------------------------------- %
% PRELIMINARIES
% ---------------------------------------------------------------------------- %
\section{Preliminaries} \label{sec:preliminaries}

\subsection{The I/O Model}
\todo[inline, caption={Preliminary: The I/O Model}]
{Model and sorting bound... \cite{Aggarwal87}}

\begin{equation*}
  \underbrace{N/B}_{\text{linear}}
  <
  \underbrace{N/B \log_{M/B} N/B}_{\text{sorting bound}}
  \ll
  N
\end{equation*}


\todo[inline, caption={Preliminary: Priority Queues}]
{Priority Queues... \cite{Arge04}}

\subsubsection{Cache-oblivious Algorithms}
\todo[inline, caption={Preliminary: Cache-oblivious Algorithms}]
{Definition of cache-oblivious algorithms and the tall cache assumption.
  \cite{Arge04}. Priority Queues are possible \cite{Arge07, Sanders2001}.}

\subsection{Ordered Boolean Decision Diagrams}
\todo[inline, caption={Preliminary: OBBDs}]
{Definition and recursive solution with memoization table \cite{Bryant86,
    Brace90, Dijk16}. Reduced OBBD's \cite[Definition2]{Bryant86}. Complementary
  edges \cite{Brace90}. Use of memoization table and motivation for I/O
  efficiency \cite{Brace90, Dijk16}.}

\subsubsection{I/O Bounds of OBDDs}
\todo[inline, caption={Preliminary: I/O OBBD Memoization issues}]
{Using a memoization table inherently will result in a lot of cache misses, when
the forest becomes large enough.}

\begin{theorem}[\cite{Arge96}] \label{thm:reduce_io_lower_bound}

  Reduction of an OBBD $G$ with minimal pair, level, depth first or breadth
  first blocking requires $\Omega(\sort(N))$ I/Os in the worst case.
\end{theorem}

\begin{theorem}[\cite{Arge96}] \label{thm:apply_io_worst_case}

  The Dynamic Programming \Apply\ algorithm on two OBDDs of size $N_1, N_2$
  followed up by a \Reduce\ operation requires $O(N_1 \cdot N_2)$ I/Os in the
  worst case.
\end{theorem}




%%% Local Variables:
%%% mode: latex
%%% TeX-master: "main"
%%% End:
